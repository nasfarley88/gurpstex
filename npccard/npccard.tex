\documentclass[draft]{article}
\usepackage{adjustbox}
\usepackage{luacode}
\usepackage{booktabs}
\usepackage{tabularx}
\usepackage{nopageno}
\usepackage{xparse}
\usepackage[utf8]{inputenc}
% \usepackage{cmbright}

\usepackage[
margin=1.2cm,
% paperwidth=5in,
% paperheight=3in
]{geometry}
\usepackage{multirow}
\usepackage{multicol}
\usepackage{array}
\usepackage{calc}
\setlength\multicolsep{-\multicolsep}

% started from http://tex.stackexchange.com/questions/7032/good-way-to-make-textcircled-numbers
\usepackage{tikz}
\usetikzlibrary{calc}
\usetikzlibrary{shapes}
\newcommand{\shapearound}[3]{\tikz[baseline=(char.base)]{
    \draw[draw=none, use as bounding box] (-1.7ex,-1.7ex) rectangle (1.7ex, 1.7ex);
    \node[#1,draw,#2,text=black,inner sep=2.7pt, anchor=center] (charmander) {};
    \node[anchor=center] (char) {#3};
  }}
\newcommand\pass[1]{\shapearound{circle}{black}{#1}}
\newcommand\critpass[1]{\shapearound{starburst, starburst point height=2pt 
    % ,text width=0.7ex, align=center
  }{black}{#1}}
\newcommand\fail[1]{\shapearound{}{white}{#1}}
% \newcommand\critfail[1]{\shapearound{single arrow, shape border rotate=270,
%     single arrow head extend=3pt}{black}{#1}} 
\newcommand\critfail[1]{\shapearound{regular polygon, regular polygon sides=8}{fill=red!20}{#1}} 

\luadirect{require("npccard.lua")}

\NewDocumentCommand{\statstack}{oooooooo}{%
  \begin{tabularx}{\linewidth}{@{}r@{}X@{}r@{}X@{}}
    % TODO make these fill up if arguments are given
    ST&\dotfill & HP&\dotfill \\
    DX&\dotfill & Will&\dotfill \\
    IQ&\dotfill & Per&\dotfill \\
    HT&\dotfill & FP&\dotfill \\
  \end{tabularx}
}

\newcommand{\nulldotfill}{\null\dotfill\\}

\newcommand{\nameaccentnotes}{Name:\dotfill \\
      Accent: \dotfill\\
      Notes: \dotfill\\
      \null\dotfill\\
      \null\dotfill\\
      \null\dotfill\\
      \null\dotfill\\
      \null\dotfill\\
      \null\dotfill\\
}

\newlength{\dicerollside}
\newlength{\dicetablecolwidth}
\newlength{\dicetablecolsep}

\newcommand{\npccard}[3]{
  % \fbox{
  % Uncomment if onecolumn
  % \begin{minipage}[b][3in][t]{0.5\linewidth}
  % Uncomment if twocolumn
  \begin{minipage}[b][3in][t]{0.95\linewidth}
    \begin{minipage}[b][2.5in][c]{0.27\linewidth}
      \textbf{Stats}\\[1ex]
      \statstack{}
      Dodge: #2 \\
      Attack to hit: #1\\
      Attack dmg: \dotfill \vfill
      \nameaccentnotes{}
    \end{minipage}
    \begin{minipage}[b][2.5in][t]{0.58\linewidth}
      \textbf{Dice rolls}\\[-1ex]
      \vspace*{-3ex}
      \begin{center}
      \tiny
        % Set length of table columns/sep for the current font
        \setlength{\dicetablecolwidth}{%
          \widthof{\luadirect{for i=1,7 do print_pass() end}}}
        \setlength{\dicetablecolsep}{%
          \widthof{\luadirect{for i=1,3 do print_pass()
              end}}-\widthof{\pass{10}}-\widthof{\pass{10}}}
        \begin{tabular}{ @{}%
          >{\raggedright\arraybackslash}p{\dicetablecolwidth}%
          @{\hspace{\dicetablecolsep}}%
          >{\raggedright\arraybackslash}p{\dicetablecolwidth}%
          @{}}%
          \textbf{Attack (#1)} & \textbf{Dodge (#2)} \\
          \luadirect{for i=1,70 do printdiceroll(#1) end}
          &%
          \luadirect{for i=1,70 do printdiceroll(#2) end}\\[\dicetablecolsep]
          \multicolumn{2}{@{}>{\raggedright\arraybackslash}p{2\dicetablecolwidth+\dicetablecolsep}@{}}{%
          \textbf{3d6}
          }\\
          \multicolumn{2}{@{}>{\raggedright\arraybackslash}p{2\dicetablecolwidth+\dicetablecolsep}@{}}{%
          \luadirect{for i=1,105 do plainprintdiceroll() end} 
          }
        \end{tabular}%
      \end{center}
    \end{minipage}
    \begin{minipage}[b][2.5in][t]{0.45in}
      \begin{center}
        \textbf{HP}\\[1ex]
        \footnotesize
        \luadirect{hpstack(#3)}
      \end{center}
    \end{minipage}\\[-2ex]
    \begin{center}
      \vspace*{-2ex}
      \begin{minipage}[b]{0.95\linewidth}
        \centering
        \textit{\luadirect{print_random_quote()}}
      \end{minipage}
    \end{center}
  \end{minipage}
% }                               % ends fbox
}

\begin{document}
\footnotesize
\noindent
% \begin{multicols}{2}
%   \noindent
%   % \npccard{attack to hit}{dodge}{max hp}
%   \npccard{11}{7}{15}\\[\fill]
%   \npccard{11}{7}{15}\\[\fill]
%   \npccard{11}{7}{15}\\[\fill]
%   \npccard{11}{7}{15}\\[\fill]
%   \npccard{11}{7}{15}\\[\fill]
%   \npccard{11}{7}{15}
% \end{multicols}

\clearpage
\tiny

\setlength{\tabcolsep}{2pt}
\begin{tabular}{cccccccccccccccccc}
  \luadirect{for i=18,2,-1 do tex.print(i .. [[ & ]]) end} 1 \\
  \luadirect{for i=18,2,-1 do printdiceroll(i) tex.print([[ & ]])end printdiceroll(1)} \\
  \luadirect{for i=18,2,-1 do printdiceroll(i) tex.print([[ & ]])end printdiceroll(1)} \\
  \luadirect{for i=18,2,-1 do printdiceroll(i) tex.print([[ & ]])end printdiceroll(1)} \\
  \luadirect{for i=18,2,-1 do printdiceroll(i) tex.print([[ & ]])end printdiceroll(1)} \\
  \luadirect{for i=18,2,-1 do printdiceroll(i) tex.print([[ & ]])end printdiceroll(1)} \\
  \luadirect{for i=18,2,-1 do printdiceroll(i) tex.print([[ & ]])end printdiceroll(1)} \\
  \luadirect{for i=18,2,-1 do printdiceroll(i) tex.print([[ & ]])end printdiceroll(1)} \\
  \luadirect{for i=18,2,-1 do printdiceroll(i) tex.print([[ & ]])end printdiceroll(1)} \\
  \luadirect{for i=18,2,-1 do printdiceroll(i) tex.print([[ & ]])end printdiceroll(1)} \\
  \luadirect{for i=18,2,-1 do printdiceroll(i) tex.print([[ & ]])end printdiceroll(1)} \\
  \luadirect{for i=18,2,-1 do printdiceroll(i) tex.print([[ & ]])end printdiceroll(1)} \\
  \luadirect{for i=18,2,-1 do printdiceroll(i) tex.print([[ & ]])end printdiceroll(1)} \\
  \luadirect{for i=18,2,-1 do printdiceroll(i) tex.print([[ & ]])end printdiceroll(1)} \\
  \luadirect{for i=18,2,-1 do printdiceroll(i) tex.print([[ & ]])end printdiceroll(1)} \\
  \luadirect{for i=18,2,-1 do printdiceroll(i) tex.print([[ & ]])end printdiceroll(1)} \\
  \luadirect{for i=18,2,-1 do printdiceroll(i) tex.print([[ & ]])end printdiceroll(1)} \\
  \luadirect{for i=18,2,-1 do printdiceroll(i) tex.print([[ & ]])end printdiceroll(1)} \\
  \luadirect{for i=18,2,-1 do printdiceroll(i) tex.print([[ & ]])end printdiceroll(1)} \\
  \luadirect{for i=18,2,-1 do printdiceroll(i) tex.print([[ & ]])end printdiceroll(1)} \\
  \luadirect{for i=18,2,-1 do printdiceroll(i) tex.print([[ & ]])end printdiceroll(1)} \\
  \luadirect{for i=18,2,-1 do printdiceroll(i) tex.print([[ & ]])end printdiceroll(1)} \\
  \luadirect{for i=18,2,-1 do printdiceroll(i) tex.print([[ & ]])end printdiceroll(1)} \\
  \luadirect{for i=18,2,-1 do printdiceroll(i) tex.print([[ & ]])end printdiceroll(1)} \\
  \luadirect{for i=18,2,-1 do printdiceroll(i) tex.print([[ & ]])end printdiceroll(1)} \\
  \luadirect{for i=18,2,-1 do printdiceroll(i) tex.print([[ & ]])end printdiceroll(1)} \\
  \luadirect{for i=18,2,-1 do printdiceroll(i) tex.print([[ & ]])end printdiceroll(1)} \\
  \luadirect{for i=18,2,-1 do printdiceroll(i) tex.print([[ & ]])end printdiceroll(1)} \\
  \luadirect{for i=18,2,-1 do printdiceroll(i) tex.print([[ & ]])end printdiceroll(1)} \\
  \luadirect{for i=18,2,-1 do printdiceroll(i) tex.print([[ & ]])end printdiceroll(1)} \\
  \luadirect{for i=18,2,-1 do printdiceroll(i) tex.print([[ & ]])end printdiceroll(1)} \\
  \luadirect{for i=18,2,-1 do printdiceroll(i) tex.print([[ & ]])end printdiceroll(1)} \\
  \luadirect{for i=18,2,-1 do printdiceroll(i) tex.print([[ & ]])end printdiceroll(1)} \\
  \luadirect{for i=18,2,-1 do printdiceroll(i) tex.print([[ & ]])end printdiceroll(1)} \\
  \luadirect{for i=18,2,-1 do printdiceroll(i) tex.print([[ & ]])end printdiceroll(1)} \\
  \luadirect{for i=18,2,-1 do printdiceroll(i) tex.print([[ & ]])end printdiceroll(1)} \\
  \luadirect{for i=18,2,-1 do printdiceroll(i) tex.print([[ & ]])end printdiceroll(1)} \\
\end{tabular}

\end{document}

%%% Local Variables:
%%% mode: latex
%%% TeX-master: t
%%% End:
